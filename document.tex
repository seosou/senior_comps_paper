\documentclass[10pt,twocolumn]{article}

% use the oxycomps style file
\usepackage{oxycomps}
\usepackage{graphicx}
\usepackage{subcaption}

% usage: \fixme[comments describing issue]{text to be fixed}
% define \fixme as not doing anything special
\newcommand{\fixme}[2][]{#2}
% overwrite it so it shows up as red
\renewcommand{\fixme}[2][]{\textcolor{red}{#2}}
% overwrite it again so related text shows as footnotes
%\renewcommand{\fixme}[2][]{\textcolor{red}{#2\footnote{#1}}}

% read references.bib for the bibtex data
\bibliography{references}

% include metadata in the generated pdf file
\pdfinfo{
    /Title (Approaching Nutrition through Recipes and Apps)
    /Author (Seolbin Hong)
}

% set the title and author information
\title{Approaching Nutrition through Recipes and Apps}
\author{Seolbin Hong}
\affiliation{Occidental College}
\email{shong4@oxy.edu}

\begin{document}

\maketitle

\section{Introduction}
Nutrition is an important aspect of health and wellness, yet many struggle to understand which foods are truly beneficial to them and how to implement healthy eating habits consistently in their daily lives. A recent study found that only around half of Americans feel highly confident in their knowledge of which foods are healthy for them.\cite{Yam_2025} This lack of confidence likely contributes to the widespread prevalence of nutritional deficiencies in the population. Common nutritional deficiencies for Americans include Vitamins A, C, D, and E, as well as Zinc, which can have significant impacts on immunity, energy, and overall physiological function .\cite{Reider_Chung_Devarshi_Grant_Hazels_Mitmesser_2020} However, interestingly enough, a 2024 survey found that in all adult age groups, improving their diet and eating healthier was one of the most popular New Year's resolutions.\cite{Good_2024} So while there is an interest in eating healthier, there is a contradiction in translating this interest into a sustainable behavior change.

There are many reasons why this is difficult. One major barrier that stops someone from implementing healthier eating is the lack of accessible, digestible nutrition education. Many people are unsure how to apply general dietary guidance to their own daily meals, especially when faced with the complexities of nutritional information and conflicting recommendations. This may lead to cognitive overload and discourage people from attempting to eat healthier, learning that nutrition is something not worth trying to understand.

Cost is another barrier, as healthier foods are often perceived, and are, more expensive. This limits options for individuals with limited budgets. There are, however, numerous websites and mobile applications that have been designed to support nutrition tracking and education. These tools often require extensive manual input, from logging meals to calculate nutrient intake, which can be time-consuming and discouraging. Many apps also cost a fee or require a subscription for users to access and use. As a result, some users do not attempt certain nutrition apps and others frequently abandon these tools, leading to low retention rates and minimal long-term impact.

To counter these issues, nutrition interventions should focus not only on providing accurate information to the users, but also communicating them in a format that is easily consumable, personalized, and actionable. By tailoring recommendations to individual preferences, dietary restrictions, and existing nutrient gaps, users can be encouraged towards choices that are more practical, engaging, and aligned with their unique needs. When combining simplified tracking mechanisms and educational resources that are able to explain the purpose and function of specific nutrients, it can create a more interactive experience to the users, leading to a potential in increasing their confidence, fostering sustainable behavior changes, and ultimately improving their overall diet.

\section{Technical Background}

People  improve their nutritional intake typically through three primary ways: increasing their nutrition knowledge, using mobile or web applications to track dietary habits, or receiving personalized dietary guidance. Mobile nutrition apps, specifically, have become widely available and many of them have high download rates, showing popularity.\cite{Nogueira-Rio_Varela} However, despite this availability, overall adoption and long-term engagement remains low.\cite{Stehr_Karnowski_Rossmann_2020}

There are different reasons to why people may not adopt these digital technologies. People are often discouraged from using these apps because of the time consumption that comes with logging in their intakes, lack of awareness of the apps' existences, and lack of nutritional literacy.\cite{ckinser@acsm.org_2022} A common limitation that I see in these tools is their static and generalized nature. Many apps are interested in documenting and summarizing users' diets, such as caloric intake or macronutrient distribution, but few actually recommend and connect viable, personalized, and dynamic next steps for users. Without this, it is difficult for users to take in such information in a meaningful way that changes their dietary habits. These apps also usually do not account for individual nutrient deficiencies, personal food preferences, dietary restrictions, cooking ability, or lifestyle constraints. This lack of personalization can reduce user engagement, and decrease their motivation to use this app as time goes on. 
However, there is a need for nutritional tools that go beyond passive data collection and instead help users take immediate, personalized action. A system that recommends recipes or meals based on a user's current nutrient profile, food preferences, and health goals could reduce cognitive load and support healthier behavior. These gaps in existing solutions motivated the design and development of the project described in this work.
\section{Prior Works}
To better understand the current approaches towards nutrition in digital technologies, I  researched some of the most popular ones. For my project, I chose to examine MyFitnessPal, WeightWatchers, and MyPlate before beginning my project. These tools primarily focus on logging and tracking dietary intake, where they often require users to enter foods manually, estimate portion sizes, or repeatedly navigate through large food databases for users to properly log in their data. Although these apps successfully document what users eat, they often do not translate this knowledge into personalized, actionable recommendations. Users are given data, but not much guidance on what to do beyond the app to improve their nutritional intake.
\subsection{MyFitnessPal}
MyFitnessPal contains a large food database for users to select from to log in their consumed items. It is primarily used to track calories and macronutrients with an emphasis in weight loss.\cite{myfitnesspal} Since, users have to manually record every meal or estimate, it requires consistent effort from the users to keep up with the app, as well as receive some benefit. However, though it records their diet, it doesn't provide dynamic actionable next steps. Also, many of its advanced features require a paid subscription, which can be a barrier for many individuals.
\subsection{WeightWatchers}
WeightWatchers is similar to MyFitnessPal, but uses its own point system to simplify when calculating calories and nutrients.\cite{weightwatchers} The points given to the user is personalized, however because the points is a created system by WeightWatchers, it hides exact nutrient values. This might be beneficial to some users, but for others, it may limit nutritional literacy and application to real life. This app also requires manual food logging, but does recommend recipes. However, these recipe recommendations are not dynamically matched to the user's nutrient deficiencies or cuisine preferences, only their point goals. This app also costs a monthly subscription. 
\subsection{MyPlate}
MyPlate is a government-backed program that is free and categorized their information by foods groups like fruits, vegetables, grains, proteins, and dairy.\cite{myplate} They can provide daily and weekly goals, which can be tailored by age, sex, weight, and activity level. It also provides basic food logging and meal tracking. However, the food logging database isn't as large as other apps and is limited to general dietary guidance with no recipe personalization. This digital application is more simple and has less features than commercial apps.

These apps show a critical gap in the current nutritional technology industry and current user needs. While data tracking is useful and well supported by these designs, personalized, actionable, and dynamic recommendations are not as well supported. Currently, there also seems to be difficulty in integrating nutrient specific guidance with user preferences and lifestyle considerations in a way that is both simple, yet engaging. This emphasizes the need for a system that combines a multi-factored recipe recommender, simplified logging, and nutrient education that is tailored to the individual.
\begin{figure*}
    \centering
    \includegraphics[height=0.33\textheight]{app_main_page.pdf}
    \caption{Main Screen}
    \label{fig:mainpage}
\end{figure*}
\section{Methods}

Based off of the understanding of these contexts and approaches, this project proposes a mobile application designed to simplify the process of understanding and improving nutritional intake. Unlike many of the existing tools that rely on extensive manual logging or general recommendations, the app focuses on easy logging and personalized, dynamic, and actionable guidance. It takes in user-provided information, like dietary preferences, restrictions, cuisine preferences, and user features, like sex, age, height, weight, and activity level, to recommend recipes that target gaps in their nutrient intake. This app also hopes to bridge the gap between nutritional knowledge, usage of the app, and application in real life by providing educational pages for each nutrient and how it functions in the body, the symptoms of surplus and deficiencies of the nutrient, as well as which foods are high and low in them.

 The idea is to help users understand not only what their current nutritional profile looks like on the daily, but also what they can do next to improve it and why it is important to. To achieve this, the app collects user-input data to curate their nutritional recommendations and their food preferences, and then these inputs are stored in Android's SharedPreferences system locally on the device. The Spoonacular API will be used as the primary database for recipes. By querying Spoonacular's database, using user's taste preferences and dietary restrictions, the app can return the recipes that best align with the user's nutritional gaps, preferences and needs.\cite{spoonacular-api} Since the app will return multiple recipes, it still allows the users the autonomy to choose which meal to consume, while still improving their diet. This approach also allows the users to learn more about their dietary habits and the types of foods they consume and offers them the opportunity to change their habits. The recipes recommended are also dynamic and adjust in real-time in response to new user input and nutritional intake. The outcome of this project is to provide actionable tools to improve user nutritional intake rather than simply providing static nutritional information. By incorporating cuisine preferences, one of the highest-rated factors influencing Americans' food choices\cite{Yam_2025}, the application aims to promote healthier, yet enjoyable eating habits. 
 Beyond recipe recommendations, the app incorporates educational elements for each nutrient, explaining its function in the body, potential symptoms of deficiency or excess, and examples of foods high or low in that nutrient. This approach aims to increase nutritional literacy, empowering users to make well-informed decisions about their diet rather than just following static meal plans. By combining personalization, simplified logging, and education, the application seeks to address the key factors, such as complexity, time consumption, and lack of actionable guidance, that have shown to hinder user uptake to healthier eating habits. The app ultimately wants to be a tool that can support and encourage sustainable improvements in dietary behavior by making nutrition accessible, engaging, and practical through a user-centered design and integration with the Spoonacular API.
 \begin{figure*}
    \centering
    \includegraphics[height=0.33\textheight, keepaspectratio]{comps_profile_page.pdf}
    \caption{Nutritional Profile}
    \label{fig:mainpage}
\end{figure*}
 \section{Evaluation Metrics}
Evaluation metrics for this project include usability, diversity of recommendations, nutrition education, and the relevance of recipe suggestions to the user's nutritional needs.

Usability was enhanced through simplified data logging. So, instead of requiring users to manually enter every ingredient and food or estimate the portion size, the application allows users to save the nutritional information of any recommended recipe with a single button. This reduces the mental load and the total amount of steps it would have taken on other nutrition apps and encourages more consistent user engagement. Another important aspect of this project is the dynamism and diversity of recipe recommendations. Unlike many existing tools, and unlike manually searching for recipes online, which often return repetitive or generic recipes, this application queries a wide array of recipes through the Spoonacular API to encourage greater adoption of diet changes rather than those that are completely different from what the user is used to. This would be done while incorporating the user's cuisine preferences and dietary restrictions. Recommendations are then ranked based on how well they address the user's specific nutritional gaps to ensure that suggestions are not only varied but also meaningfully aligned with the user's overall nutritional profile and not just similar or redundant recipe recipes. Another part of the evaluation focuses on nutritional education, which was the reason that many dietitians used their nutrition apps for, educating their patients.\cite{Nogueira-Rio_Varela} The app provides dedicated pages for each nutrient, including a clear definition of its physiological role, symptoms associated with deficiencies and surpluses, and examples of foods that are naturally high or low in that nutrient. This educational aspect is included to support nutritional literacy, helping users understand why certain nutrients matter and how specific foods contribute to their health. It also provides them the autonomy to fill in the gaps of their nutritional intake beyond the recipes in the apps. They can then add or subtract from their nutritional profiles manually as well if they choose to do so. By integrating actionable recommendations with accessible nutritional education, the app seeks to improve both user behavior and nutritional understanding. 
Several other evaluation metrics were considered during the design of this project, but were ultimately not implemented due to the constraints in the scope, time, and data availability. One metric was long-term health impact, where the app measures whether or not users improve their nutrient intake or overall health over an extended period of time. While this would be valuable information about the app's real-world impact, conducting an evaluation like this would have required a long data collection that did not fit the scope of this project. 
Another metric that could have been explored is the algorithmic accuracy in recipe recommendation and how well it aligns to the user's nutritional deficits and preferences, however it would have been difficult to measure and optimize given the time constraint, so I chose to focus more on the foundational aspects of the project, while leaving this to be something to consider for future applications. 
There was also consideration for comparing performances between this app and mainstream search engines. For example, comparing whether or not this app could provide more diverse or nutritionally targeted/user-relevant recipe suggestions than Google (or other tools). One of the motivations for this app was due to my frustration of wanting to learn more about what I was consuming and finding relevant recipes, if not ingredient substitutions to better balance my diet and fill in the gaps of my nutritional intake. However when I used a search engine, I would either find generic sets of recipes or ones that did not fit all of my needs. Researching about nutrition was also difficult and required many steps and some time. If I were to have compared that with this app, it would have been difficult to define which consistent inputs to use because the search engine and this app were both designed differently and have different processes that they were meant to support. Also, the Spoonacular API, while extensive and fine for this comprehensive project, had limitations in its dataset and the completeness in some of the recipes' nutritional information as some recipes were given by  users, removing consistency and completeness.\cite{spoonacular-api} This would not allow for a robust, fair comparison. Thus, these data and systematic constraints would have made it difficult to compare and so this evaluation was not pursued. 
Other potential metrics that might seem applicable to a nutrition app include user retention, dietary logging frequency, and behavioral engagement. However, they were not pursued due to the limitation in providing a device for the users to have over time periods, as well as the time restrictions of this project. 
Conclusively, the evaluations used in this project were the ones that could best be done with the resources available. This project evaluated usability, diversity of recommendations, relevance to user nutritional needs, and overall intuitiveness of navigating the app. I believed that these metrics best allowed the assessment of whether the app addressed the goals of reducing cognitive load, simplifying nutrition tracking, and encouraging users towards more nutritionally beneficial choices.

\subsection{Results and Discussion}

Through iterative informal user testing, several consistent themes emerged regarding how users interacted with the application. Across all sessions, users expressed greater inclination to extend their use of the app, and actual adoption in real life, when the experience felt personalized to them. Users were able to understand the nutrient descriptions and how those nutrients related to their personal health. Many also mentioned that the interface was intuitive and easy to navigate, though in the beginning rounds of user testing it was not.

Through this testing, personalization was identified as the key driver of sustained engagement and user satisfaction. Users responded more positively to the app when it had the ability to incorporate their cuisine preferences, dietary restrictions, and nutritional gaps into recipe recommendations. Because the app felt more personalized, some said that suggestions felt more compelling and actionable. They also liked that when they had the information on each nutrient, they had the autonomy to fill in the gaps of certain nutrients with fruits or vegetables when necessary and had the ability to edit their nutritional profile directly. It allows for accuracy and flexibility, especially if dietary habits change, supplements are taken, or when adjustments are necessary. Participants that had experience with nutrition logging apps expressed that the manual food logging felt tedious and discouraging, but this app helped remove those extra steps. Users said that this would make them more willing to maintain consistent logs, as the process did not feel as time-consuming. In the later iterations of user testing, participants expressed greater contentment of the app when they saw how dynamic the recipes were and that they saw recipes they would actually cook and/or have eaten in the past. They also liked that the recipes and nutritional profile updated in response to one another and in real time. They also appreciated the ease of logging in their nutritional intake with the use of a single button. For users that have had experience with other nutrition apps, they found this to be less time consuming and less mentally draining.

Nutritional literacy helped increase the perceived usefulness and credibility of the app. Users reported that they felt more confident in making dietary decisions when they had access to clear explanations of each nutrient, including its purpose in the body, symptoms of deficiency or surplus, and examples of foods high and low in that nutrient. I would assume that this is the case, but it brings up an interesting conversation because this study has found that apps can provide contradictory statements when discussing differences in certain features like age or education.\cite{Stehr_Karnowski_Rossmann_2020} However, they also felt more confident in making dietary decisions when they knew how their nutritional intake was calculated and that it was in agreement with national and international health organizations. Even though some did not interact with the app for long, they felt more curious about their  nutritional intake and felt that they had learned from the app due to the educational pages. This educational component helped bridge a common gap observed in existing tools. Though many apps present nutritional data, they rarely contextualize its meaning so the educational pages supported both engagement and informed decision-making. 

The testing also revealed that ease of navigation was essential for maintaining user interest and curiosity. In the beginning of testing, when participants had difficulty moving around the app, they felt dissatisfied and disengaged from the app and its impact. For example, some users reported that it was unclear how to change their nutritional logs or find the user profile, which made their experience feel clunky and limited their curiosity and willingness to explore the app further. This lack of intuitive design created a barrier not only to usability but also to the motivation that they felt. They were less likey to consistely track their nutrition when they encountered repeated navigation difficulties. However, after improving from these issues by making clearer and more intuitive lables, the app was able to have a more logical flow between the screens, which users have expressed greater curiosity and engagement. They were more willing to explore the features and the purpose of the app, such as viewing the different nutritional information, as well as experimenting with the recipe recommendations. After reducing the complexity and confusion of the app's interface, it decreased the cognitive load, and allowed space and room for the users to learn about their nutritional profile rather than how to use the app. 
This goes to show the importance of user-centered design in health application. When I was able to observe real user interactions and improve the interface, it supported the idea that intuitive navigation can directly help user retention, as well as learning. Through this process, I found that users who could quickly and easily find the features felt a greater sense of control and interest in what the app had to offer. 
Overall, the results indicate that an application focused on reduced logging, increased personalization, and access to nutritional literature can meaningfully support user engagement. 

\subsection{Limitations and Future Considerations}

While the application shows some success in demonstrating a personalized recipe recommendation system for improving diet, there are several limitations that I would like to acknowledge. The scope of this project is limited. It focuses primarily on generating recipe recommendations and providing basic nutritional education. Thus, some features commonly found in commercial nutrition apps like grocery list generators, meal planners and ability to share with others are not included.\cite{ckinser@acsm.org_2022} Without these features, it limits the app's ability to support users through all the actions that go into cooking, such as planning, shopping, and tracking meals. Though the app incorporates nutritional gaps, cuisine preferences, and dietary restrictions, it is not as complex or personalized as it could be. It does not consider cooking time, recipe complexity, equipment availability, costs and ingredient overlap between recipes into its recommendations, which are important factors when deciding to cook. Thus, this provides room for improvement and brings reason to why the recommendations may not always align with user practicality, budget, or lifestyle and instead be rendered useless because it does not meet user's needs. This can impact the usability of the app along with its retention. The search function is also limited due to how Spoonacular interprets it. When a user uses the search function, Spoonacular takes it as a "tag" and tries its best to match the request, but if it does not have a match, it either returns no recipes or the next best matches that may not be of user's needs or preferences.\cite{spoonacular-api} The data limitation in Spoonacular also provides room for improvement. When testing for a vegan diet, there were limited recipes for certain cuisine types and overall, is not as expansive as other apps.

While the app reduces the burden of manual logging each and every ingredient and food item, it still lacks a comprehensive food search feature or barcode scanner to be able to integrate outside food into the user's nutritional profile. This feature can be added through Spoonacular, however due to the restrictions of the time and scope of the project, this was not pursued. If it was, this would have made logging in snacks, drinks, and restaurant meals easier for the users, and provide a more accurate and flexible experience for their nutritional profile. 

Additionally, the app is not compatible with users inputting their own recipes or recipes from other sources. This was a question raised by some of the participants in the study, but was not pursued for this project because of the scope and time limitations. The app also does not have a personal recipe log for users to look back to. This restricts users' ability to look back on dishes they might have wanted to prepare regularly, for example due to their dietary habits or lifestyle.

Finally, while the user interface is functional and is intuitive to a certain extent, it is still relatively minimal and could be improved with better design choices or interactive visualizations to help improve clarity, ease of features and improve engagement with the app. This would help to increase possible retention and user satisfaction. The project was mainly focused on proving its core functionality, but with more time and research, the user interface is something that I would have liked to be improved.

So, for the future direction of this project, to improve the usefulness, personalization, and practicality of the application, I would have included a grocery list generator to help make the app more engaging and encouraging to use. I would have converted the chosen recipes into a consolidated shopping list so that the user could have a complete workflow from identifying their nutritional gaps to choosing their desired recipes to acquiring the ingredients necessary to address them. It could be helpful to provide prices to the ingredient items and also assistance to find the more unique, cultural ingredients. I would also definitely develop a more robust recipe recommendation system. I would test around with the values more to help balance the nutritional profiles more with the test case that the ideal user saves 3 meals from the app. I would also make this more intuitive by allowing the user to enter how many meals they typically consume to help with their nutritional intake. I would have also incorporated factors such as cost, preparation time, ingredient overlap, equipment availability and user cooking experience into the recommendation system. Incorporating similar ingredient based recommendations may also help reduce food waste and lower grocery expenses by suggesting recipes that reuse ingredients already purchased or left over. I would also look into how users feel about cooking so that the app can feel more supportive and intuitive to their experience.

Due to the data limitation of Spoonacular, I would want to implement a way for users to import personal recipes or manually enter ingredients for their recipes to better increase the app's adaptability. This feature could support more cultural dishes, family recipes, or meals not found in the Spoonacular database and enhance the app's overall inclusivity and cultural awareness. There is also room for the app to include a recipe history or log, enabling users to track recipes they have tried, liked, or saved for later. This could support long-term habit formation and understanding and allow for a more robust recommending system that learns from user behavior. This could also show the user what types of ingredients they lean towards and if there are any ingredient substitutions they can take instead, in regards to their nutritional profiles. For example, if they are eating sandwiches often and eating white bread, they could opt for whole grain bread.

Refining the user interface is always an improvement that I would want to make, as it is an important aspect for retainability and user experience. A more visually appealing and intuitive user interface, with clearer navigation, nutrient visualizations, and interactive elements would further improve usability and user satisfaction. There could also be animations, progress tracking, and visual nutrient summaries that could provide a more engaging and educational experience. All these additions could provide a more comprehensive, user-centered nutrition tool that is capable of supporting diverse user needs in the long-term.

\section{Ethical Considerations}

The project was built with careful consideration to the potential ethical implications of nutritional tracking and recommendation technologies. One main concern that I had with this project was that it would inadvertently contribute to disordered eating behaviors. Although the app emphasizes balanced nutrition and educational guidance, there remains a possibility that some users may over-rely on logging or recommendations in a way that is unhealthy. 
The nutritional intake data used in this project was drawn from publicly available databases, which are limited in scope. Notably, there is a lack of comprehensive study on nutritional needs for intersex, non-gender conforming individuals, and other underrepresented populations.\cite{Linsenmeyer_2024} As a result, the recommendations provided by the app may not fully capture the diversity of nutritional requirements across different sex identities, gender identities, life stages, or medical conditions.\cite{Stehr_Karnowski_Rossmann_2020} 
One of the features to calculate nutritional need was activity level, but that can be difficult to quantify. The U.S. Department of Agriculture also includes activity level in their DRI calculator, but they do not explain how to quantify the different levels.\cite{DRI} There could be more done to provide more transparency in nutrition and how it works for different bodies. 
Data and privacy are also important to mention for this project. Although this app stores user data locally using Android's SharedPreferences rather than sending it to external servers, users should be aware that personal dietary and health information is sensitive information. Some apps today take user information and make profit off of them by selling them to other companies. Even if they have a privacy policy, it may be vague or difficult to understand, so users may not be giving fully informed consent. Some users might feel that it's the cost to use the app and allow for their data to be sold. If this app were to grow and required cloud storage, then it should obey the privacy regulations and best practices, so that data is securely stored, encrypted, and provide users with transparency on how the data will be collected, used, and/or kept. Maintaining trust and protecting user data is essential for all apps, but especially for nutrition and health apps. 
This app was intended as an additional educational tool rather than a replacement for existing healthcare infrastructure. So, while it can provide insights and convenience, there is a broader concern that technological creations may encourage "solutionism" in public health, emphasizing technological fixes rather than addressing systemic issues such as healthcare accessibility, food security, and public nutrition education. Economic and educational disparities further complicate the ethical dilemma regarding nutrition and people's access to them. People with limited financial resources or lower nutritional literacy may find it more difficult to act on recommendations, either due to cost constraints or limited access to healthy foods.\cite{Yam_2025} These factors contribute to the unequal opportunities for engagement with apps like this one and reinforce the need for awareness of social determinants of health. 
So, while this project hoped to improve nutrition awareness and accessibility, it fully acknowledges its limitations and broader ethical implications. Users are advised to treat the app as an informative tool rather than a definitive authority, where it is strongly encouraged to consult healthcare professionals for personalized guidance as nutrition is highly individual and are influenced by genetics, lifestyle, nutritional goals and health conditions. With that, developers should also remain cognizant of the potential societal, psychological, and equity-related consequences of digital nutrition tracking and educational technologies.

\section{Replication Instructions}

\subsection{Hardware Requirements and Downloads}

There are no hardware requirements to use the Spoonacular API. However, Android Studio has specific RAM, storage, and CPU requirements. These can be found in the official Android Studio installation documentation \cite{android-install}. This project was built using Android Studio version Narwhal 3 (September 2025). This specific version can be downloaded from the Android Studio archive \cite{android-archive}. The latest version of Android Studio is available on their download page \cite{android-latest}. To run the application, you will also need a Spoonacular account to obtain an API key \cite{spoonacular-api}.

\subsection{How to Run the Project}

To run the project, create a Spoonacular account and access your API Key through "My Console" then "Profile \& API Key". Download Android Studio and the repository\cite{repository}. Then, unzip the repository to get senior\_comps-master folder. Open Android Studio, select "Open" and choose the senior\_comps-master folder. After opening the project, navigate to the strings.xml file to put inyour API Key where indicated. 

If your computer meets the hardware requirements for the Android Emulator, press the "Run" button (the arrow icon at the top of Android Studio). The emulator will launch automatically, and the application will appear on the simulated device.

To run the application on a physical device, enable Developer Options and USB Debugging on your Android device. Connect the device to your computer using USB or Wi-Fi. Once connected, click the "Run" button, and the app will run on your device. Additional details about device setup can be found in the official documentation \cite{android-device}.

\subsection{Code Architecture Overview}

The main components of this project include Adapters, Listeners, Models, activity java files, and XML files. The data for the user are stored in Android's SharedPreferences and the recipes are retrieved from the Spoonacular API. The nutritional need for the user is calculated with the Mifflin-St Jeor equation and the USDA/WHO nutritional recommendations. The algorithm to calculate which recipes will best help fill in the gaps in the user's nutritional profile is  simple. It takes in an arbitrary base value for each nutrient, then a weighted value based on nutritional deficiency/surplus (which can either help the score or penalize it), after, the top 5 recipes are shown to the user unless it is extremely unhealthy. Currently, the code does not include desserts from Spoonacular or caters towards users that have certain fitness goals in mind like muscle gain or weight loss. There can also be improvements in the user interface to help users navigate the app easier and there is also no current log of recipes that the user can access to see the recipes that they saved the nutritional information of.

If you add another java activity file that is not a fragment or composable, you will have to put them in Android.XML or the IDE will throw an exception error.

\printbibliography

\appendix

\end{document}
